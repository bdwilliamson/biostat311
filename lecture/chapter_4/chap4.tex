\documentclass[12pt, 
hyperref={colorlinks=true, linkcolor=blue, urlcolor=cyan}]{beamer}
\usetheme{default} 

\setbeamertemplate{navigation symbols}{} %gets rid of navigation symbols
\setbeamertemplate{footline}{} %gets rid of bottom navigation bars
\setbeamertemplate{footline}[page number]{} %use this for page numbers

\setbeamertemplate{itemize items}[circle] %round bullet points
\setlength\parskip{10pt} % white space between paragraphs

\usepackage{wrapfig}
\usepackage{subfig}
\usepackage{setspace}
\usepackage{enumerate}
\usepackage{graphicx}
\usepackage{amsmath}
\usepackage{amsfonts}
\usepackage{amssymb}
\usepackage{amsthm}
\usepackage[UKenglish]{isodate}
\cleanlookdateon

% the preamble
\title{BIOST 311: \\ Regression Methods for the Health Sciences}
\author{Kelsey Grinde and Brian Williamson}
\institute{UW Biostatistics}
\date{Spring 2018}

\begin{document}
% title slide
\begin{frame}
\titlepage\thispagestyle{empty}
\end{frame}

% make it 4.something
\setbeamertemplate{footline}{%
  \raisebox{5pt}{\makebox[\paperwidth]{\makebox[120pt]{\scriptsize Last updated \today}\hfill\makebox[10pt]{\scriptsize 4.\insertframenumber~~}}}}  \newcounter{chap1}{\value{1}}
\setcounter{framenumber}{\value{chap1}}

\begin{frame}
\frametitle{CHAPTER 4: SPECIAL TOPICS}
At the end of this chapter, the typical student should be able to:
\begin{itemize}
\item do something!
\end{itemize}
\end{frame}

\begin{frame}
\frametitle{SECTION 1: {\small NONPARAMETRIC ESTIMATION AND INFERENCE}}
\framesubtitle{(by Brian Williamson)}

So far, in this course, you have explored how to answer scientific questions related to: \vspace{-0.3cm}
\begin{itemize}
\item causal associations,
\item associations (and prediction),
\item and effect modification,
\end{itemize} \vspace{-0.3cm}
using linear regression, logistic regression, or Cox proportional hazards regression.

You have also explored the necessary assumptions for these models to be valid.

Fundamentally, however, these tools \textcolor{red}{all make potentially restrictive assumptions} on the true data-generating mechanism (i.e., the process that truly creates the data as we see it).
\end{frame}

% introduce the variable importance problem, in the context of HIV vaccine
\begin{frame}
\frametitle{Case study: variable importance}

\end{frame}

% how would we address this using linear regression?
\begin{frame}
\frametitle{Case study: variable importance}

\end{frame}

% the model as a collection of probability distributions
\begin{frame}
\frametitle{Statistical models}
\end{frame}

% the parameter as a map from the collection to the real line
\begin{frame}
\frametitle{Statistical parameters: more than just $\beta$s?}
\end{frame}

% regression coefficients as a nonparametric parameter
\begin{frame}
\frametitle{Statistical parameters: linear regression}

\end{frame}

% procedure that separates statistical model and parameter from estimation
\begin{frame}
\frametitle{New paradigm}
\end{frame}

% variable importance parameter
\begin{frame}
\frametitle{Case study: variable importance}
\end{frame}

% how to estimate it?





\end{document}
