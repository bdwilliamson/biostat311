\documentclass[12pt, 
hyperref={colorlinks=true, linkcolor=blue, urlcolor=cyan},dvipsnames]{beamer}
\usetheme{default} 

\setbeamertemplate{navigation symbols}{} %gets rid of navigation symbols
\setbeamertemplate{footline}{} %gets rid of bottom navigation bars
\setbeamertemplate{footline}[page number]{} %use this for page numbers

\setbeamertemplate{itemize items}[circle] %round bullet points
\setlength\parskip{10pt} % white space between paragraphs

\usepackage{wrapfig}
\usepackage{subfig}
\usepackage{setspace}
\usepackage{enumerate}
\usepackage{graphicx}
\usepackage{amsmath}
\usepackage{amsfonts}
\usepackage{amssymb}
\usepackage{amsthm}
\usepackage[UKenglish]{isodate}
\usepackage{verbatim}
\usepackage{xcolor}
\cleanlookdateon

% new amber color
\definecolor{amber}{rgb}{1.0, 0.75, 0.0}

\DeclareMathOperator{\argmin}{argmin}

% the preamble
\title{BIOST 311: \\ Regression Methods for the Health Sciences}
\author{Kelsey Grinde and Brian Williamson}
\institute{UW Biostatistics}
\date{Spring 2018}

\begin{document}
% title slide
\begin{frame}
\titlepage\thispagestyle{empty}
\end{frame}

% make it 1.something
\setbeamertemplate{footline}{%
  \raisebox{5pt}{\makebox[\paperwidth]{\makebox[120pt]{\scriptsize Last updated \today}\hfill\makebox[20pt]{\scriptsize 3.\insertframenumber~~}}}}  \newcounter{chap3}{\value{1}}
\setcounter{framenumber}{\value{chap3}}

\begin{frame}
\frametitle{CHAPTER 3: SURVIVAL ANALYSIS}
By the end of Chapter 3, you should be able to: \vspace{-0.3cm}

\begin{itemize}
\item Determine if a variable has been \textcolor{BurntOrange}{right-censored}
\item Discuss the \textcolor{red}{drawbacks} of treating a right-censored variable as binary or continuous
\item \textbf{Interpret Kaplan-Meier curves}, and create them in \texttt{R}
\item \textbf{Implement and interpret} a log-rank test for equating survival curves
\item \textbf{Formulate a regression model} given a scientific or statistical question about a right-censored outcome
\item \textbf{Interpret the coefficients} for a (simple or multiple) proportional hazards regression model
\item \textbf{Interpret confidence intervals and p-values} for proportional hazards regression coefficients
\item Use \texttt{R} to fit a proportional hazards regression model and create figures/tables to support your regression analysis
\end{itemize}

\end{frame}

\section{Censored outcomes}
\begin{frame}
\frametitle{SECTION 1: CENSORED OUTCOMES}
Up to this point, we have focused on questions involving \textcolor{blue}{quantitative} or \textcolor{BurntOrange}{binary} outcomes: 
\begin{itemize}
\item Is lung function (\textcolor{blue}{FEV}) associated with smoking, after adjusting for age, height, and sex?
\item Is cognitive function (\textcolor{blue}{DSST score}) associated with alcohol use, adjusting for age, sex, and education?
\item Is \textcolor{BurntOrange}{diabetes} associated with BMI, after adjusting for sex?
\end{itemize}
\end{frame}

\begin{frame}
\frametitle{SECTION 1: CENSORED OUTCOMES}
However, we are often interested in scientific questions that involve \textcolor{blue}{time-to-event} outcomes:
\begin{itemize}
\item Is \textcolor{blue}{time to first seizure post operation to remove a brain tumor} associated with pre-operation case review by an epileptologist, in children with both epilepsy and brain tumor?
\item Is \textcolor{blue}{time to promotion for university faculty members} associated with sex?
\item Is \textcolor{blue}{time to death from any cause} associated with serum levels of C reactive protein?
\end{itemize}
\end{frame}

\begin{frame}
\frametitle{Characteristics of survival data}
We wish to study whether \textcolor{blue}{meditation} prolongs the \textbf{time until a severe panic attack} in patients suffering from a panic disorder.

To address this question, we: \vspace{-0.3cm}
\begin{itemize}
\item recruit 200 patients and randomize them to meditation or placebo;
\item follow each patient until their first severe panic attach post recruitment;
\item and compare the mean time until a severe panic attack in each group with a t-test
\end{itemize}

{\fontsize{10pt}{7.2}\selectfont
Sample of the observed times until first severe panic attack (in weeks):
\hspace*{-1cm}\begin{tabular}{|c|c|c|c|c|c|c|c|c|c|}
\hline
\textcolor{blue}{meditation group} & 24.29 & 46.81 & 40.23 & 27.18 & 10.56 & 88.5 & 53.57 & 8.81  \\
\hline
control group & 30.2 & 5.94 & 58.75 & 9.54 & 47.89 & 16.55 & 28.29 & 4.58 \\
\hline
\end{tabular}
}
\end{frame}

\begin{frame}
\frametitle{Characteristics of survival data}
\centering
\includegraphics[width=0.8\textwidth]{figs/meditation_observed_study_time.png}
\end{frame}

\begin{frame}
\frametitle{Characteristics of survival data}
\centering
\includegraphics[width=0.8\textwidth]{figs/meditation_observed_rand_time.png}
\end{frame}


\begin{comment}
\section{Nonparametric estimation and inference}
\begin{frame}
\frametitle{SECTION 2: NONPARAMETRIC ESTIMATION}
\end{frame}

\section{Semiparametric estimation and inference}
\begin{frame}
\frametitle{SECTION 3: SEMIPARAMETRIC ESTIMATION}
\end{frame}
\section{Summary}
\begin{frame}
\frametitle{SECTION 4: SUMMARY}
\end{frame}
\end{comment}
\end{document}