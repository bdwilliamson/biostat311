\documentclass[12pt, 
hyperref={colorlinks=true, linkcolor=blue, urlcolor=cyan},dvipsnames]{beamer}
\usetheme{default} 

\setbeamertemplate{navigation symbols}{} %gets rid of navigation symbols
\setbeamertemplate{footline}{} %gets rid of bottom navigation bars
\setbeamertemplate{footline}[page number]{} %use this for page numbers

\setbeamertemplate{itemize items}[circle] %round bullet points
\setlength\parskip{10pt} % white space between paragraphs

\usepackage{wrapfig}
\usepackage{subfig}
\usepackage{setspace}
\usepackage{enumerate}
\usepackage{graphicx}
\usepackage{amsmath}
\usepackage{amsfonts}
\usepackage{amssymb}
\usepackage{amsthm}
\usepackage[UKenglish]{isodate}
\usepackage{verbatim}
\usepackage{xcolor}
\cleanlookdateon

% new amber color
\definecolor{amber}{rgb}{1.0, 0.75, 0.0}

\DeclareMathOperator{\argmin}{argmin}

% the preamble
\title{BIOST 311: \\ Regression Methods for the Health Sciences}
\author{Kelsey Grinde and Brian Williamson}
\institute{UW Biostatistics}
\date{Spring 2018}

\begin{document}
% title slide
\begin{frame}
\titlepage\thispagestyle{empty}
\end{frame}

% make it 1.something
\setbeamertemplate{footline}{%
  \raisebox{5pt}{\makebox[\paperwidth]{\makebox[120pt]{\scriptsize Last updated \today}\hfill\makebox[20pt]{\scriptsize 3.\insertframenumber~~}}}}  \newcounter{chap3}{\value{1}}
\setcounter{framenumber}{\value{chap3}}

\begin{frame}
\frametitle{CHAPTER 3: SURVIVAL ANALYSIS}
By the end of Chapter 3, you should be able to: \vspace{-0.3cm}

\begin{itemize}
\item Determine if a variable has been \textcolor{BurntOrange}{right- or left-censored}
\item Discuss the \textcolor{blue}{benefits} and \textcolor{red}{drawbacks} of treating a right- or left-censored variable as binary or continuous
\item Create a \texttt{Survival} object in \texttt{R}
\item \textbf{Interpret Kaplan-Meier curves}, and create them in \texttt{R}
\item \textbf{Formulate a regression model} given a scientific or statistical question about a right- or left-censored outcome
\item \textbf{Interpret the coefficients} for a (simple or multiple) proportional hazards regression model
\item \textbf{Interpret confidence intervals and p-values} for proportional hazards regression coefficients
\item Use \texttt{R} to fit a proportional hazards regression model and create figures/tables to support your regression analysis
\end{itemize}

\end{frame}

\end{document}