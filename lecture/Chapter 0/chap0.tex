\documentclass[12pt, 
hyperref={colorlinks=true, linkcolor=blue, urlcolor=cyan}]{beamer}
\usetheme{default} 

\setbeamertemplate{navigation symbols}{} %gets rid of navigation symbols
\setbeamertemplate{footline}{} %gets rid of bottom navigation bars
\setbeamertemplate{footline}[page number]{} %use this for page numbers

\setbeamertemplate{itemize items}[circle] %round bullet points
\setlength\parskip{10pt} % white space between paragraphs

\usepackage{wrapfig}
\usepackage{subfig}
\usepackage{setspace}
\usepackage{enumerate}
\usepackage{graphicx}
\usepackage{amsmath}
\usepackage{amsfonts}
\usepackage{amssymb}
\usepackage{amsthm}
\usepackage[UKenglish]{isodate}
\usepackage{color}
\cleanlookdateon

% the preamble
\title{BIOST 311: \\ Regression Methods for the Health Sciences}
\author{Kelsey Grinde and Brian Williamson}
\institute{UW Biostatistics}
\date{Spring 2018}
\begin{document}
% title slide
\begin{frame}
\titlepage\thispagestyle{empty}
\end{frame}

% do de-brief discusion section, mention index card grade, check that everyone has been able to knit, reminder about OH and HW
\begin{frame}
\frametitle{Checking in...}
\begin{itemize}
\item Yesterday's discussion section % everyone has submitted; don't worry about submission time -- just by end of day; grading is just credit/no credit; congrats on surviving day 1: everyone did REALLY well -- we're very impressed by how quickly you picked things up; we'll continue to work on this throughout the quarter, so it might feel like a lot right now but it will get easier!
	\begin{itemize}
	\item Congrats on creating your first R Mardown document!
	\item FYI: grading is credit/no credit
	\end{itemize}
\item Grades for index card 1 are posted % everyone has 100%! moving forward: grading is always credit/no credit; about once per week
	\begin{itemize}
	\item Future index cards/activities $\approx$ once/week
	\item FYI: grading is credit/no credit
	\end{itemize}
\item Starting your \href{https://canvas.uw.edu/courses/1203588/assignments/4098142}{first assignment} % can you access? can you knit? after today, should be able to do everything; please don't save knitting until the last minute!
	\begin{itemize}
	\item Download the \texttt{.Rmd} file
	\item Update the author, output type, type in your answers % (should be able to answer everything after class today)
	\item \texttt{Knit} (try this early!)
	\item Turn in .pdf/Word doc that gets created \textit{and} \texttt{.Rmd} file
	\end{itemize}
\item Questions? Come to office hours today (2--4) % check if time works for most people
\end{itemize}
\end{frame}


% make it 0.something
\setbeamertemplate{footline}{%
  \raisebox{5pt}{\makebox[\paperwidth]{\makebox[120pt]{\scriptsize Last updated \today}\hfill\makebox[10pt]{\scriptsize 0.\insertframenumber~~}}}}
\newcounter{chap0}{\value{1}}
\setcounter{framenumber}{\value{chap0}}

% Chapter 0 learning objectives
\begin{frame} 
\frametitle{CHAPTER 0: BIOST 310 REVIEW}
By the end of Chapter 0 you should be able to: \vspace{-0.2cm}
\begin{itemize}
\item Identify and describe common study designs
\item Classify variables according to their type (e.g., binary)
\item Use numerical and graphical tools to summarize data
\item Explain features of and calculate probabilities using important probability distributions (Normal, t)
\item Define a sampling distribution
\item State the Central Limit Theorem and understand its importance for statistical inference
\item Construct and interpret confidence intervals
\item Formulate null and alternative hypothesis
\item Choose, implement, and interpret hypothesis tests
\end{itemize}
\end{frame}

% Chapter 0 motivation
\begin{frame}
\frametitle{Why spend time reviewing these concepts?} % different courses, different instructors (and refresh for those for whom it's been awhile)
1. So everyone starts on the same page \\
2. These concepts are \textit{very} important for data analysis: \vspace{-0.35cm} % key components of data analysis
	\begin{itemize} \small \itemsep -0.5pt
	\item Understanding the data you have and its limititations: how (\textit{study design}) and what  (\textit{classifying variables}) was collected
	\item Converting your scientific questions into statistical ones (\textit{formulating hypotheses})
	\item Choosing appropriate statistical methods (based on $\uparrow$) 
	\item Summarizing your data (\textit{numerical and graphical summaries})
	\item \begin{footnotesize} (Most minor of all: running your chosen statistical methods) \end{footnotesize}
	\item Correctly interpreting your statistical results (\textit{parameter estimates, confidence intervals, p-values})
	\end{itemize}
\end{frame}

% Study design motivation
\begin{frame}
\frametitle{Study Design}
\textit{How you collect data impacts what questions you can (cannot) answer, what statistical methods you can (cannot) use, and what conclusions you can (cannot) draw.}

\color{blue} Extreme Example: \color{black} Suppose I'm interested in understanding public opinion about biostatistics. I  randomly select one individual from our class and ask if he/she likes biostatistics. \vspace{-0.3cm}

\begin{itemize} % this example is a bit contrived, but demonstrates some of why we care so much about study design
\item What questions can I answer using these data? 
\item What statistical methods can I use to analyze these data? 
\item What conclusions can I draw about public opinion of biostatstics using these data? 
\end{itemize} 

\end{frame}

% types of study designs
\begin{frame}
\frametitle{Study Design: Experimental vs Observational}

\color{blue} Experimental: \color{black} exposure/treatment is \textit{controlled} by the researcher (e.g., we randomly assign people to drug or placebo) \vspace{-0.3cm}
	\begin{itemize}
	\item Randomized controlled trial % aka randomized control trial, randomized clinical trial
	\end{itemize}

\color{blue} Observational: \color{black} exposure/treatment is \textit{not contrlled} by the researcher (e.g., we look at a group of people and \textit{observe} who smokes and who doesn't smoke)  \vspace{-0.3cm}
	\begin{itemize}
	\item Cross-sectional study
	\item Cohort study
	\item Case-control study
	\end{itemize}
\end{frame}

\begin{frame}
\frametitle{Study Design: Experimental vs Observational}

In \textit{experimental} studies, we can talk about \color{blue} \textit{causation}\color{black}. In \textit{observational} studies we have to talk instead about \color{blue} \textit{association} \color{black}(because we worry about \color{blue} \textit{confounding}\color{black}). % we'll come back to this when we talk about interpreting results

A \color{blue} \textit{confounder} \color{black} is a variable that is causally associated with our outcome and also associated with the exposure in our sample. \vspace{-0.2cm}
\begin{itemize}
\item[] For example: suppose we're interested in the relationship between smoking and lung function in kids. We know that age is causally associated with lung function: as you get older, your lung function improves. If age is also associatd with smoking in our sample (i.e., if older kids are more likely to smoke), then age is a confounder.
\end{itemize} % lots more on confounding to come, stay tuned!

\end{frame}

\begin{frame} % RCT
\frametitle{Experimental studies: randomized controlled trial}
Description:\vspace{-0.3cm}
\begin{itemize}
\item Take a sample of the population, \textbf{assign} to treatment (\textit{exposed}) or control/placebo (\textit{unexposed}), and follow for disease outcomes (death yes/no, time to death)
\end{itemize}

Pros:\vspace{-0.3cm}
\begin{itemize}
\item With a large enough sample, no confounding!
\item Gold standard for causality
\end{itemize}

Cons:\vspace{-0.3cm}
\begin{itemize}
\item Often very expensive
\item Not always possible or ethical to randomize
	\begin{itemize}
	\item Can't randomize sex, race, age, genetic variants
	\item Unethical to randomize to harmful exposures
	\end{itemize}
\end{itemize}
\end{frame}

\begin{frame}
\frametitle{Experimental studies: randomized controlled trial}
Examples:
\begin{itemize}
\item \href{https://jamanetwork.com/journals/jama/article-abstract/2613159}{Effect of Vitamin D and Calcium Supplementation on Cancer Incidence in Older Women}
\item \href{http://stroke.ahajournals.org/content/36/8/1764.short}{Daily Functioning and Quality of Life in a Randomized Controlled Trial of Therapeutic Exercise for Subacute Stroke Survivors}
\end{itemize}
\end{frame}

% cross-sectional studies
\begin{frame}
\frametitle{Observational studies: cross-sectional study}
Description: \vspace{-0.3cm}
\begin{itemize}
\item Randomly sample individuals, record exposure and outcome at a \textit{single time point} (no follow-up) % take a snapshot
\end{itemize}

Pros:\vspace{-0.3cm}
\begin{itemize}
\item Relatively cheap and easy
\item Can study multiple outcomes and exposures
\end{itemize}

Cons:\vspace{-0.3cm}
\begin{itemize}
\item Inefficient for rare exposure or disease % you won't have very many (or any) people in your sample with the thing you're interested in
\item Time sequence of exposure and outcome (i.e., which came first) is not always clear % did low exercise lead to obesity, or obesity lead to low exercise?
\item Potential confounding
\end{itemize}
\end{frame}

\begin{frame}
\frametitle{Observational studies: cross-sectional study}
Examples:
\begin{itemize}
\item \href{http://ajph.aphapublications.org/doi/abs/10.2105/AJPH.78.10.1336}{Job strain, work place social support, and cardiovascular disease in a random sample of the Swedish working population}
\item \href{onlinelibrary.wiley.com/doi/10.1111/add.12623/full}{Real-world effectiveness of e-cigarettes when used to aid smoking cessation}
\end{itemize}
\end{frame}


% cohort studies
\begin{frame}
\frametitle{Observational studies: cohort study}
Description: \vspace{-0.3cm}
\begin{itemize}
\item Sample \textit{people without the outcome}, measure their exposure, then \textit{follow} those individuals and see who gets outcome % often called longitudinal cohort study
\end{itemize}

Pros:\vspace{-0.3cm}
\begin{itemize}
\item Time sequence is known (exposure came first)
\item Can study multiple outcomes 
\end{itemize}

Cons:\vspace{-0.3cm}
\begin{itemize}
\item Inefficient for rare outcomes % have to wait a lont time to get enough people with disease
\item Often expensive and time-consuming % and more time for people to drop out
\item Potential confounding
\end{itemize}
\end{frame}

\begin{frame}
\frametitle{Observational studies: cohort study}
Examples:
\begin{itemize}
\item \href{https://www.medicalnewstoday.com/articles/316619.php}{Drinking tea could help stave off cognitive decline}
\item \href{https://www.medicalnewstoday.com/articles/316565.php}{Birth control pills may protect against some cancers for decades}
\end{itemize}
\end{frame}

% case-control studies
\begin{frame}
\frametitle{Observational studies: case-control study}
Description: \vspace{-0.3cm}
\begin{itemize}
\item Sample individuals with and without the disease \textit{(based on outcome)}, look back in time (usually) for exposure status % when you look back in time = retrospective case-control study
\end{itemize}

Pros:\vspace{-0.3cm}
\begin{itemize}
\item Efficient for rare diseases % (don't have to recruit as many people)
\item Cheaper and faster than cohort studies
\item Can study multiple exposures
\end{itemize}

Cons:\vspace{-0.3cm}
\begin{itemize}
\item May not know time sequence of disease and exposure
\item Cannot use to estimate relative risk or disease prevalence
\item Potential confounding
\end{itemize}
\end{frame}

\begin{frame}
\frametitle{Observational studies: case-control study}
Examples:
\begin{itemize}
\item \href{https://www.sciencedirect.com/science/article/pii/S0140673605676635}{Obesity and the risk of myocardial infarction in 27,000 participants from 52 countries}
\item \href{http://www.nejm.org/doi/full/10.1056/NEJMoa065497\#t=article}{Case control study of human papillomavirus and oropharyngeal cancer}
\end{itemize}
\end{frame}

% let's practice
\begin{frame}
\frametitle{Study Design: Let's practice!}

Recent (June 2017) \href{http://www.nejm.org/doi/10.1056/NEJMoa1702747}{air pollution study} in the news: \vspace{-0.3cm}
\begin{itemize} % headlines
\item ``U.S. air pollution still kills thousands every year, study concludes" -- NPR
\item ``Even 'safe' pollution levels can be deadly" --- NYT
\end{itemize}

From the article: \textit{We conducted a nationwide ... study involving all Medicare benficiaries from 2000 through 2012, a population of 61 million, with 460 million person-years of follow-up.}

\color{blue} What kind of study design is this? Why do you think they chose this design? What are potential limitations of this study design? \color{black} % (think $\rightarrow$ pair $\rightarrow$ share). NOTE: this is the type of thing you'll need to be able to do on HW/exams
\end{frame}

% types of variables (brief reminders before we get to summarizing data)
\begin{frame}
\frametitle{Types of Variables} % why do we care

\textit{How you summarize/analyze your data often depends on what type of data you've collected.}

\color{blue} Extreme Example: \color{black} Suppose I'm intereseted in summarizing one of the variables we collected in our FEV dataset: smoking habits (smoker/nonsmoker).  What numerical summaries can I use to get a sense of ``typical" smoking habits in our study? \vspace{-0.3cm}
\begin{itemize}
\item Mean/average
\item Median
\item Mode
\item Proportion in each category
\end{itemize}
\end{frame}

\begin{frame}
\frametitle{Types of Variables}

\color{blue} Categorical: \color{black} limited number of possible values
\begin{itemize}
\item \textit{Nominal:} order doesn't matter (e.g., blood type)
\item \textit{Ordinal:} order does matter (e.g., level of education)
\item \textit{Binary:} two possible values; nominal or ordinal (e.g., sex)
\end{itemize}

\color{blue} Quantitative: \color{black} infinite possible values
\begin{itemize}
\item \textit{Discrete:} values are integers; often counts (e.g., \#  teeth)
\item \textit{Continuous:} values on continuum (e.g., height)
\end{itemize}

\end{frame}

\begin{frame} % air pollution example
\frametitle{Types of Variables: Let's practice!}

From \href{http://www.nejm.org/doi/10.1056/NEJMoa1702747}{``Air pollution and mortality in the medicare population."}

Some variables they collected:
\begin{itemize}
\item Death/mortality
\item Daily PM$_{2.5}$ concentration (micrograms per cubic meter)
\item Age
\item Race 
\end{itemize}

\color{blue} Classify these variables according to their type. \color{black}

\end{frame}

% summarizing data
\begin{frame} % why do we care?
\frametitle{Summarizing Data}

\textit{Why is it so important to know how to summarize data? Because we can't look at all of it, but we still want to make sense of it/draw conclusions from it.}

\color{blue} Extreme example: \color{black} The Human Genome

\center \includegraphics[height=4cm]{./dna}

\end{frame}

\begin{frame}
\frametitle{Summarizing Data: The Human Genome}

A single human DNA sequence consists of over 3 billion A's, C's, T's, and G's. 

If we were to write out one sequence in \begin{tiny} tiny \end{tiny} font, what would that look like? \vspace{-0.3cm}

\center \includegraphics[height=4cm]{./book}

How many pages will it take? How many books?
\end{frame}

\begin{frame}
\frametitle{Summarizing Data: The Human Genome}

\vspace{-0.4cm}
\center \includegraphics[height=0.65\textheight, angle = 270]{./bookcase}

\end{frame}

\begin{frame}
\frametitle{Summarizing Data}

Why do we need descriptive statistics?\vspace{-0.2cm}
\begin{itemize}
\item Making sense of large amounts of data
\item Checking data quality
	\begin{itemize}
	\item Values outside reasonable range (e.g., height = nine feet, age = 200)
	\item Implausible combinations of variables (e.g., pregnant men)
	\end{itemize}
\item Observe distribution of variables in dataset
	\begin{itemize}
	\item Measures of center and spread
	%\item Skewness
	\end{itemize}
\item Start to understand direction and strength of association
	\begin{itemize}
	\item Descriptive statistics and inferential analysis should contribute to the same story
	\end{itemize}
\end{itemize}

\end{frame}

% types of descriptive statistics
\begin{frame}
\frametitle{Descriptive Statistics: Outline}

\begin{itemize}
\item Univariate
	\begin{itemize}
	\item Numerical (categorical variables)
	\item Numerical (quantitative variables)
	\item Graphical
	\end{itemize}
\item Stratified
\item Bivariate (mostly graphical)
\end{itemize}

\end{frame}

% numerical descrip for categorical variables
\begin{frame}
\frametitle{Univariate Descriptive Statistics: Categorical}

Summary should include:
\begin{itemize}
\item Number and percent in each group
\item If binary, only need to summarize one group (other can be inferred)
\item Also good to summarize missing data (e.g., \#  and proportion missing values)
\end{itemize}

% look at Table 1 in air pollution paper for example: what's there, what's missing (# and # missing)
\end{frame}

% numerical descrip for quantitative variables
\begin{frame}
\frametitle{Univariate Descriptive Statistics: Quantitative}

Summary should include (some of):
\begin{itemize}
\item Sample size
\item Number of missing observations
\item Measures of center: 
	\begin{itemize}
	\item Sample mean
	\item Sample median
	\end{itemize}
\item Measures of spread:
	\begin{itemize}
	\item Sample standard deviation
	\item Interquartile range (IQR): (Q1,Q3) or (Q3 - Q1)
	\item Range: (Min,Max) or (Max - Min)
	\end{itemize}
%\item Note: you can detect skewness by comparing mean to median, IQR to median
%\item Note: you can detect outliers by looking at range
\end{itemize}
% look at Table 1 in air pollution paper for example: what's there, what's missing (spread, missing)
\end{frame}

% histograms
\begin{frame}
\frametitle{Univariate Descriptive Statistics: Graphical}

Histograms are useful for describing the shape of a distribution:\vspace{-0.8cm}

\center \includegraphics[height=0.7\textheight]{./histogram-age}

\vspace{-0.5cm} \begin{scriptsize} R code:  \texttt{hist(fev\$age, xlab="Age (years)", ylab="Frequency", main="Histogram of Age")} \end{scriptsize}

\end{frame}

% boxplots
\begin{frame}
\frametitle{Univariate Descriptive Statistics: Graphical}

Boxplots are useful for seeing outliers and central location:\vspace{-0.8cm}

\center \includegraphics[height=0.7\textheight]{./boxplot-age}

\vspace{-1cm} \begin{scriptsize} \texttt{boxplot(fev\$age, xlab="", ylab="Age (years)", main="Boxplot of Age")} \end{scriptsize}

\end{frame}

% barplots
\begin{frame}
\frametitle{Univariate Descriptive Statistics: Graphical}

Barplots can be used to summarize categorical variables: \\ \begin{footnotesize} (but often don't provide much/any more info than numerical summary) \end{footnotesize}\vspace{-0.8cm}

\center \includegraphics[height=0.7\textheight]{./barplot-smoke}

\vspace{-0.8cm} \begin{scriptsize} \texttt{barplot(table(fev\$smoke), xlab='Smoking (1 = yes, 2 = no)', ylab='Count', main='Bar Plot of Smoking')} \end{scriptsize}
\end{frame}

% stratified descriptive statistics
\begin{frame}
\frametitle{Stratified Descriptive Statistics} % useful for categorical vs quantitative comparisons

Rather than looking at the distribution of a variable across the entire dataset, we might be interested in the distribution of that variable within certain subgroups defined by another variable. \vspace{-0.3cm}
\begin{itemize}
\item e.g., how does the distribution of age differ between smokers and non-smokers in our FEV dataset?
\item e.g., how does the distribution of FEV between smokers and non-smokers?
\end{itemize}

Stratified descriptive statistics can help us:\vspace{-0.3cm}
\begin{itemize}
\item understand the role of that  stratification variable (e.g., is it a confounder?)
\item begin to demonstrate the association between two variables
\end{itemize}

% examples in air pollution Table 1
\end{frame}

% stratified histograms
\begin{frame}
\frametitle{Stratified Descriptive Statistics}

\begin{center} \includegraphics[width=\textwidth]{./histogram-age-stratified} \end{center}

\end{frame}

% stratified boxplots
\begin{frame}
\frametitle{Stratified Descriptive Statistics}

\vspace{-0.4cm} \center \includegraphics[height=0.6\textheight]{./boxplot-age-stratified}

\begin{scriptsize} \texttt{boxplot(fev\$age~fev\$smoke, xlab='Smoking (1=yes, 2=no)', ylab='Age (years)', main = 'Boxplot of Age by Smoking')} \end{scriptsize}

\end{frame}

% scatterplot
\begin{frame}
\frametitle{Bivariate Descriptive Statistics}

A useful tool for summarizing the relationship between two quantitative variables:

\vspace{-0.4cm} \center \includegraphics[height=0.6\textheight]{./scatterplot-fev-age}

\begin{scriptsize} \texttt{plot(fev\$fev ~ fev\$age, xlab='Age (years)', ylab='FEV (liters per sec)', main = 'Scatterplot of FEV vs Age')
} \end{scriptsize}

\end{frame}

% descrip in R
\begin{frame}
\frametitle{Descriptive Statistics in R}

This is just a small taste of the many possibilities when it comes to tools for summarizing data.

We'll spend more time discussing descriptive statistics in Discussion Section next week, including how to generate numerical summaries and figures like these (or better ones!). 
\end{frame}

% practice: air pollution (can continue to next lecture if needed); move to beginning of descriptives section?
\begin{frame}
\frametitle{Descriptive Statistics: Let's Practice!}

Read parts of the air pollution and mortality paper: Results on page 2515, Table 1, and Figure 1

Answer the following questions:
\begin{itemize}
\item What numerical summaries are presented for categorical variables? For continuous variables?
\item What numerical summaries are missing for categorical variables? For continuous variables?
\item Are any stratified descriptive statistics presented? If so, why do you think they presented these?
\item Are any graphical summaries presented? If so, why do you think they presented these?
\end{itemize}
% pair > share
\end{frame}

% What's next?
\begin{frame}
\frametitle{What's Next?}
\begin{itemize}
\item Probability distributions: Normal, $t$
\item Sampling distributions
\item Central Limit Theorem
\item Confidence intervals
\item Hypothesis testing
\end{itemize}
\end{frame}


\end{document}
